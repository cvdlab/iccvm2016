\section{Introduction}\label{introduction}

An \emph{interactive indoor mapping} environment consists of a virtual reconstruction
of a physical indoor space, in which the user can move around and interact
with virtual objects, that are found in the same position they actually occupy in the
real world. Such an interactive indoor mapping can be thought as a specialized
and very evoluted \emph{user interface} capable of giving a glimpse of a section of
the real world that the user can handle in a natural and intuitive way. Such
a reconstructed virtual indoor environment can be considered a general
platform where many different applications can rely upon. Both promising and
already well explored ICT applications may find in \emph{virtual indoor mapping} the
perfect context to be integrated into.

In particular, for environments with massive presence of sensor-equipped (or
``smart'') objects, which realize the so-called \emph{IoT} (Internet of
Things), the interactive indoor mapping represents an ideal integrated
interface for IoT monitoring systems. To be specific, it can be the container
of indoor navigation systems, giving the user, to be routed across an indoor
environment, the opportunity to interact with objects along the suggested
paths. Furthermore, in conjunction with the advancements in the field of user
indoor location, whose efforts are nowadays focused to realize an integration
of positioning systems like GNSS (Global Navigation Satellite system), Wi-Fi,
Bluetooth and LTE (Long Term Evolution), to support continuous outdoor/indoor
navigation by means of integration of technologies, it represents the most
natural interface to perform realtime access monitoring and multi-person
tracking.

To enable such an interactive mapping platform it is of the utmost importance
to set up  a descriptive representation of the indoor environment. 
 This description belongs to  the field of indoor cartography,
which as digital evolution of plain floor plans, has arrived to arouse the
interest of big players like Google, that has integrated indoor plans of
specific locations of interest \cite{indoormaps} into Google Maps. In general,
it can be considered ``of interest'' --- such to justify and motivate indoor
cartographic applications --- both public or commercial places of vast
dimensions, as for example airports, train stations, shopping malls, and also
private buildings subject to strict access protocols, like warehouses,
logistic centers, data centers, etc.

The kind of evoluted user interface outlined above is provided by the
\emph{FIVE} Web Framework, whose design choices and implementation details
represent the main contribution of this paper, alongside with the definition
of the HIJSON document format, which responds to the need of a descriptive
representation. A comprehensive toolkit to process the document format is also
introduced.

Moreover, this paper quickly outlines the generation of geometric data of a
complex building, to provide both an explicit semantic and a hierarchical
model of indoor spaces. \textsc{\large lar}, a general representation for
geometric and solid modeling is used for this purpose. The generated
\textsc{\large lar} structures are exported to \textsc{\large hijson} format,
extending \textsc{\large geojson} for indoor mapping and the Internet-of-
Things. A convenient way to extend the representation capabilities of IoT
\emph{smart objects} is also mentioned as semantic extensions, that affects
both document format and the web framework, and can be easily collected in a
public repository.


% This work, jointly developed by Sogei S.p.A., an ICT company fully owned by
% Italian  Ministry of Economy and Finance, and the CVDLAB (Computational Visual
% Design Laboratory) of the ``Roma Tre'' University, is inspired by the
% necessities of Sogei itself, which runs one of the largest data center of
% Europe, so requiring very strict access control policies, which include the
% recording and the real-time interaction with man/machine maintenance
% scenarios. Support for this interactive framework, where realtime awareness of
% the maintainer position inside the data center helps to reduce intervention
% times and to increase safety and security, has been chosen as case study of
% interactive indoor mapping, based on the proposed indoor cartographic format.


% Anonymised version  
This work, jointly developed by Xxxxxxxxxxxx, an ICT
company fully owned by Xxxxxxx xxxxxxx of Xxxxxxx and Xxxxxxx, and the YYYYYY
(Yyyyyyyyyyyyy Yyyyyy Yyyyyy Yyyyyyyyyy) of the ``Yyyy Yyy'' University, is
inspired by the necessities of Xxxxx itself, which runs one of the largest
data center of Europe, so requiring very strict access control policies, which
include the recording and the real-time interaction with man/machine
maintenance scenarios. Support for this interactive framework, where realtime
awareness of the maintainer position inside the data center helps to reduce
intervention times and to increase safety and security, represents a valid
case study of interactive indoor mapping.

% Support for this interactive framework, where realtime awareness of the
% maintainer position inside the data center helps to reduce intervention times
% and to increase safety and security, has been chosen as case study of
% interactive indoor mapping, based on the proposed indoor cartographic format.


The remainder of this document is organized as follows. In Section~\ref
{related-work} we provide an overview of the state of the art in the field of
indoor document standards and related applications. Section~\ref{framework} is
devoted to present the FIVE Web Framework, focusing on its architecture and
supported applications, while Section~\ref{hijson} introduces the HIJSON
format specifically defined to describe indoor environments.
Section~\ref{toolkit} reports about the software toolkit developed to handle
the new document format. In Section~\ref{workflow} it is depicted the
operative workflow adopted to realize the mapping of an hospital. Finally,
Section~\ref{conclusions} proposes some conclusive remarks and future
developments.


% \begin{figure}[!h]
%  \centering
%  \begin{subfigure}[b]{\linewidth}
%  \includegraphics[width=\textwidth]{images/minimum-data}
%  \end{subfigure}
% \\
%  \begin{subfigure}[b]{0.48\linewidth}
%  \includegraphics[width=\textwidth]{images/minimum-colors-a}
%  \caption{}
%  \vspace*{4mm}
%  \end{subfigure}
%  ~
%  \begin{subfigure}[b]{0.48\linewidth}
%  \includegraphics[width=\textwidth]{images/minimum-colors-b}
%  \caption{}
%  \vspace*{4mm}
%  \end{subfigure}
% \\
%  \begin{subfigure}[b]{0.74\linewidth}
%  \centering
%  \includegraphics[width=\textwidth]{images/boundary}
%  \caption{}
%  \end{subfigure}
 
%  \caption{A toy example of the LAR scheme: (a) the bare minimum of data with \emph{complete} information about topology; (b) the extracted boundary; (c) the extraction method $[e] = [\partial][f]$ giving the coordinate representation (in the discrete basis of the 1-cells) ofthe boundary edges $[e]$ by product of the sparse boundary operator matrix $[\partial]$ times the coordinate representation $[f]$ of the 2-cells (faces), in the discrete basis of the 2-cells.}
%  \label{fig:minimum-data}
% \end{figure}

% -----------------------------------------------------------------------------


% VASTI AMBIENTI 
% MOLTI SENSORI E SISTEMI DI RILEVAZIONE DELLA POSIZIONE DIFFERENTI
% CONTINUOUS OUTDOOR-INDOOR NAVIGATION
% MONITORAGGIO DI SCENARI UNIFICATI DI INTERVENTO UOMO-MACCHINA
% PERSONALE TECNICO DEVE ESSERE GUIDATO 
% IL PASSAGGIO DA SERVER A IOT È IMMEDIATO
% MONITORAGGIO UNIFICATO DEGLI SMART OBJECTS IOT
% un ambiente indoor virtuale è l'ambiente perfetto per andare a monitorare l'IoT.

% IN QUESTO SCENARIO SI PONE IL LAVORO DESCRITTO IN QUESTO PAPER, 
% IN CUI LE NECESSIT`A DESCRITTE VENGONO AFFRONTATE PARTENDO DALLE BASI,
% OVVERO DEFINENDO UN FORMATO DI DOCUMENTO PER LA DESCRIZIONE ASTRATTA DI AMBIENTI INDOOR.
% DESCRITTO L'AMBIENTE DEVE QUINDI ESSERE POSSIBILE RICOSTRUIRE VIRTUALMENTE L'AMBIENTE
% RENDERLO LARGAMENTE ACCESSIBILE VIA WEB, E MONTARE SU QUESTA RAPPRESENTAZIONE VIRTUALE 
% LA POSSIBILIT`A DI INTERAGIRE CON GLI OGGETTI ALL'INTERNO DELL'AMBIENTE. L'INTERAZIONE DEVE CONSISTERE DA UN LATO NALLA POSSIBILIT`A DI RICEVERE INFORMAZIONI DALL'OGGETTO, MA ANCHE DI INVIARE COMANDI ALL'OGGETTP.

% SI REALIZZA IN QUESTO MODO UNO SCENARIO IN CUI UN SUPERVISORE INTERAGISCE CON L'AMBIENTE IN CUI SI MUOVE UN EXPLORER (O MANUTENTORE) POTENDO IL SUPERVISORE AVERE IMMEDIATA NOTIFICA DELLA POSIZIONE DEL MANUTENTORE, E AVENDO UN QUADRO COMPLETO FORNITO DAGLI OGGETTI SMART PRESENTI NELL'AMBIENTE REALE ASSIEME ALL'EXPLORER.

% D'ALTRA PARTE PER GRANDI SPAZI ANCHE L'EXPLORER PUO ESSERE SUPPORTATO DAL SISTEMA CHE AVENDO COMPLETA CONOSCENZA DELLA TOPOLOGIA E DELLA GEOMETRIA DELL'AMBIENTE, NONCHE DEGLI OGGETTI IN ESSO CONTENUTI, POSSIEDE TUTTE LE INFORMAZIONI NECESSARIE PER GUIDARE L'EXPLORER ATTRAVERSO L'AMBIENTE.

% NULLA IMPEDISCE DI MIXARE LE NECESSITÀ DI EXPLORER E SUPERVISOR, REALIZZANDO UN EXPLORER CHE NAVIGA NELL'AMBIENTE REALE ED IN ESSO RICEVE INFORMAZIONE E A CONTEMPO PUÒ INVIARE COMANDI AGLI SMART OBJECT INTORNO A SE.
