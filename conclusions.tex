\section{Conclusions}\label{conclusions}

We have introduced here some computational tools and a novel file format 
for virtual interactive indoor mapping.
Utilization of local metric coordinate system,
avoiding the manipulation of global geographical coordinates, really inconvenient when
dealing with indoor spaces and objects, greatly enhances the modeling and rendering
of the document content. 

The HIJSON format focuses on a algebraic representation of the indoor
spaces that allows for completely capturing their topology. On the basis of
this representation a virtual web environment can be rebuilt working as a
unifying platform to run a bunch of different applications. The reference
architecture of such a platform has been also implemented and described in
this work. 

The architecture supports a whole range of applications: IoT monitoring,
realtime multi-person tracking and user cross-storey navigation are already
implemented and described. A very convenient way to extend the representation
capabilities of smart objects is also mentioned as semantic extensions. These
extensions, which affects both document format and its web framework, might be
easily collected in a public repository. Community could both use public
available extensions or contribute by mapping new (smart) objects inside the
HIJSON document format.
