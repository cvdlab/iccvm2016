\section{Conclusions}\label{conclusions}

In this paper a novel document format, named HIJSON, for indoor cartographical descriptions has been
introduced. Utilization of local metric coordinate system,
avoiding the manipulation of global geographical coordinates, really inconvenient when
dealing with indoor spaces and objects, greatly enhances the modeling and rendering
of the document content. Currently, we produce the HIJSON document from a python
script using two libraries for geometric computing (\texttt{pyplasm} and \texttt{larcc} \cite{Dicarlo:2014:TNL:2543138.2543294,paoluzziMS:2014,cadanda:2015}).
The modeling process can be further improved by implementing a LAR-based 
graphical editor to assist the user during the description of the indoor
space. The realization of such an editor is already in our plans.

The HIJSON format focuses on a hierarchical representation of the indoor
spaces that allows for completely capturing their topology. On the basis of
this representation a virtual web environment can be rebuilt working as a
unifying platform to run a bunch of different applications. The reference
architecture of such a platform has been also implemented and described in
this work. 

The architecture supports a whole range of applications: IoT monitoring,
realtime multi-person tracking and user cross-storey navigation are already
implemented and described. A very convenient way to extend the representation
capabilities of smart objects is also mentioned as semantic extensions. These
extensions, which affects both document format and its web framework, might be
easily collected in a public repository. Community could both use public
available extensions or contribute by mapping new (smart) objects inside the
HIJSON document format.
