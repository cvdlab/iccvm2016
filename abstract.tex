\begin{abstract}

This paper\footnote{This work was partially
supported by grant 2014/15 from Sogei S.p.A., the ICT company of the Italian Ministry of Economy and Finance.} introduces \textsc{\large five} (Framework for Indoor Virtual Environments) and \textsc{\large hijson} (Hierarchical Interactive JSON), respectively a web toolkit for indoor mapping applications and a novel cartographic document format. \textsc{\large five} applications are entirely based on web technologies and rely on \textsc{\large hijson} documents which are in turn processed by a specialized software toolkit. An operative workflow for automated HIJSON documents production using the \textsc{\large lar} representation scheme for topology and geometry, is also outlined.

An \emph{interactive indoor mapping} environment is a virtual reconstruction of a physical indoor space, where the user may interact with virtual objects, experienced in the actual position they occupy in the real world. Our approach outlines a specialized and evoluted  3D \emph{user interface} giving a glimpse of a section of the real world, that the user can handle intuitively.  Furthermore, the virtual indoor environment API provides a platform where many different applications can rely upon. Accessible via web browsers from any kind of device, several applications may coexist on this platform. IoT monitoring, realtime multi-person tracking, and  cross-storey user navigation, are already implemented using an automatic search for all valid walkable routes, and taking into account both architectural obstacles and furniture.

The \textsc{\large hijson} format is used to represent any geometry of the indoor space of complex buildings, capturing their hierarchical structure, a complete representation of their topology, and all the objects (either ``smart'' or not) contained inside. Such textual representation allows the \textsc{\large five} framework to offer a web environment in which the user is presented with 2D or 3D models to navigate. With respect to current cartographic formats, \textsc{\large hijson} introduces four major enhancements: (a) exposes a hierarchical structure; (b) uses local metric coordinate systems; (c) may import external geometric models; (d) accepts semantic extensions.
These semantic extensions encapsulate the details about communication protocols, rendering style, and exchanged and displayed information, allowing the  format to be extended with any sort of models of objects, sensors or behaviors.

\end{abstract}
